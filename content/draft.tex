% vim: textwidth=80
The topic for this masters thesis is \enquote{Event Based Storage of Geospatial
Vector Data}. The questions asked are:

\begin{itemize}
\item How does a model for storing event based geospatial vector data look
  like?
\item What are the differences from the traditional \enquote{snapshot} model?
\item How do we measure the benefits of the storage footprint of this data?  
\end{itemize}
Geospatial vector data \todo[noline, size=\tiny]{define \enquote{geospatial
vector data}} has traditionallybeen distributed as snapshot datasets. This means
that the data is distributed as complete datasets between two versions of the
data, even though only a small part of the data has changed.

% \todo[inline]{Assume that we are using an event based model of distributing
% data. What are the differences depending on the perspectives of
% \begin{itemize}
%   \item a map data distributor
%   \item a map maker
%   \item a map service provider
%   \item a map service consumer
% \end{itemize}}


\section{From Physical to Digital Maps}
Maps are important tools for effective communication about geographic space and
topology. Early humans probably used simple representations of the world around
them, drawn in the dirt, to communicate with their peers on the African svannah
about where the best hunting grounds were, and where they could find water, food
and shelter. Not least, it would be important to describe to your friends where
you last saw a dangerous predator, like a lion. Such representations are what we
today would call maps. Maps are physical representations of the
geographic world, and lets us connect our internal metaphysical
models of the world (cognitive map) with the physical world, and communicate
them with other humans.

Humans are storytelling people. A popular historian, Yuval N. Harari, says that
our ability to tell stories, both fictional and factual, are one of the most
important attributes that set us apart from other animals during our cognitive
evolution, more than \todo{fact-check} 40.000 years ago. This ability to tell and
believe in stories made it possible for large groups of humans to cooperate, and
build the highly organized society we live in, in the modern day. The map has
undoubtedly been one of the most important tools for this cooperation.

With the establishment of information science as a field of study by scholars
such as Harry Nyquist and Ralph Hartley during the 20s, and expanded upon by
Claude Shannon during the 40s, and the development of computer technology during
the 50s and 60s, \acrshort{gisc} saw its first explorations during this time.
Roger Tomlinson, a Canadian geographer, is counted by many as one of the
pioneers in this field of study.

The 70s saw further advanecment of hardware and software made for spatial
analysis, and in the 80s the personal computer proliferated and made
computers with \acrshort{gis} more accessible.

In the 90s \acrshort{gps} and remote sensing became more integrated with
\acrshort{gis}, and accurate and real-time data collection became a thing.
The internet played a crutial role in the dissemination of geographic
information, and started the democratization of geographic data. One important
service in this regard is \acrshort{osm}.

The 2000s saw the proliferation of web-based mapping services, and the use of
\acrshort{gis} in fields such as epidemiology, transportation planning etc.
The use of \acrshort{gis} has become widespread in government, business and
academia.

Today, \acrlong{ai}, big data analysis and augmented reality are large topics in
\acrshort{gisc} research.



% NOTE:
% • Snapshot vs event based
% • History of gis
% • History of time in geospatial information science
% • Change ontology
% • Change deteciton
% • Temporal models
% • Spatio-temporal queries
% • Worboys' stages of spatio-temporal gis
% • Event based
% • Data distributer vs data consumer
% • Who digitalized cartography


% Maps are great tools for effective communication about geographic space and
% topology.  The first graphic forms of communication can be traced back to about
% forty thousand years ago \autocite{Harley87}, in the Late Stone Age (Upper
% Paleolithic), and represents important historical evidence of a cognitive
% revolution happening in the human biology. Graphic representation of geographic
% space in the form of maps connects the internal metaphysical models of human
% minds (cognitive map) with the physical world. This connection enables
% effective communication and reasoning about complex spatial and topological
% structures and relationships. Hence, maps have for thousands of years been an
% important catalyst for social cooperation, so much that it has become the topic
% of the science that today is know as \emph{cartography}.
%
% Humans are storytelling animals, and for millennia people have been
% communicating narratives by using natural language. The stories humans tell
% each other contains descriptions of \enquote{things} and \enquote{happenings}.
% More importantly, human stories describes the interaction between
% \enquote{things} and \enquote{happenings} by depicting an ordered sequence of
% causes and effects, constituting what we call \emph{causality}. The development
% of causal reasoning is said to be the most significant advance in the evolution
% of human cognition and is essential for ability to perceive time
% \autocite{StuartFox15}. Causal cognition enables reasoning about events in the
% past, as well as the future, which makes it possible to do strategic planning.
% Thus, telling stories by using natural language enables even more effective
% ways of cooperation because it lets humans share knowledge between individuals.
%
%
% While maps are good for communicating geospatial topology, they are 
% % TODO: Structural language and computer maps %
%
% Although the world we live in is very dynamic, topics in \acrshort{gis} and
% \acrfull{gisc} research are largely concerned static representations of the
% world. Well known conceptual data models such as \emph{raster} and
% \emph{vector} formats are used to model spatial data in geographic space
% (\emph{geospatial} data) irrespective of the existence of a temporal dimension
% in the application data domain.
%
% Humans have been drawing geographic maps for thousands of years 
%
% Temporality is often neglected, because there is no need to represent time in a
% static world view. This is unsurprising when w
%
% \section{Movement Representation and Modeling}
%
% \subsection{Movement in Geographic Space}
%
% % TODO: What is movement? %
% Movement is a subcategory of \emph{change}, and can be modeled as a function of
% time.\mymarginpar{spatio–temporal change} Specifically, movement is the change
% of an objects spatial relations to other objects over time, i.e., movement is a
% \emph{spatio–temporal} phenomenon. The emphasis in this work is on
% \emph{geographical space}, meaning that that movement is discussed about
% objects that are spatially related to the Earth, i.e. \emph{geospatial}
% objects.
%
% %To elaborate, geographical space can be modelled by using geographic coordinate
% %systems, where coordinates are often provided as tuples of numbers or symbols.
% %For instance, a spherical (or ellipsoidal coordinate system) might provide
% %coordinates as tuples of latitude and longitude; a map projection system might
% %describe coordinates projected on a plane as $x$ and $y$ values; an
% %earth–centered, earth–fixed (ECEF) system might describe coordinates as
% %Cartesian coordinates in three dimensional space; and a geocode system might
% %describe geographic entities using a single sequence of numbers or symbols,
% %e.g. a postal code
%
% Although reality is highly dynamic, models and representations in
% \acrfullpl{gis} and \acrfull{gisc} are dominated by static views of
% geographic phenomena. In the literature this
%
% \textcite{Peuquet2001} explains that there are two basic types of change
% relating to objects. The first type is change of location in space, i.e.
% \emph{translation}. For instance the movement of a vehicle, pedestrian, or an
% animal.  The second type is changing spatial extent, e.g. by growing, shrinking
% or changing shape. Point objects have no spatial extent, and are thus only
% subject to the first type of change, while higher–dimensional geometries can
% exercise both types of change, either as discrete events or happening at the
% same time. Examples are storm–systems, vegetation areas and land use.
% % "dynamic" vs "static"
% In addition, \textcite{Goodchild2007} presents a third dimension of
% \todo{elaborate on Goodchild's third dimension of temporal variability}
%
%
% While both types of changes can be perceived as
% \enquote{movement}, only the former will be considered in the following.
% Furthermore, the emphasis is constrained to movement of zero-dimensional
% points, or \acrfullpl{mpo}.
%
% Considering an Euclidean vector space, the notion of movement as described
% above, can be modeled as a specific type of Euclidean (or rigid) transformation
% — namely \emph{translation}. Modeling translation in two dimension as a
% function of time, can be written as:
%
% \begin{equation}\label{eq:translation}
%   (x_0,y_0) \rightarrow (x_0 + a_t, y_0 + b_t)\text{,}
% \end{equation}
%
% where $(x_0, y_0)$ are the original coordinates, and $(a_t, b_t)$ is the translation vector. \Cref{eq:translation} could also be written using vector notation:
%
% \begin{equation}\label{eq:trans_vector}
%   \boldsymbol{x_0} \rightarrow \boldsymbol{x_0} + \boldsymbol{v_t}
% \end{equation}
%
% Here, $\boldsymbol{v_t} = (a_t, b_t)$. \Cref{eq:translation,eq:trans_vector} are
%
% A more formal definition can be borrowed from the mathematical field of geometry.
%
% % Peuquet2001: Two basic types of change: Change in location in space, Change in
% % spatial extent (growing, shrinking, changing shape).
%
% % Transformations of the Euclidean plane (Worboys2004, ch. 3)
