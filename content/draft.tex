The topic for this masters thesis is \enquote{Event Based Storage of Geospatial
Vector Data}. The questions asked are:

\begin{itemize}
\item How does a model for storing event based geospatial vector data look
  like?
\item What are the differences from the traditional \enquote{snapshot} model?
\item How do we measure the benefits of the storage footprint of this data?  
\end{itemize}
Geospatial vector data \todo[noline, size=\tiny]{define \enquote{geospatial vector data}} has
traditionallybeen distributed as snapshot datasets. This means that the data
is distributed as complete datasets between two versions of the data, even
though only a small part of the data has changed.

% NOTE:
% • Snapshot vs event based
% • History of gis
% • History of time in geospatial information science
% • Change ontology
% • Change deteciton
% • Temporal models
% • Spatio-temporal queries
% • Worboys' stages of spatio-temporal gis
% • Event based
% • Data distributer vs data consumer
% • Who digitalized cartography

\section{From Physical to Digital Maps}
Maps are important tools for effective communication about geographic space and
topology. Early humans probably used simple representations of the world around
them, drawn in the dirt, to communicate with their peers on the African svannah
about where the best hunting grounds were, and where they could find water, food
and shelter. Not least, it would be important to describe to your friends where
you last saw a dangerous predator, like a lion. Such representations are what we
today would call maps. Maps are physical representations of the
geographic world, and lets us connect our internal metaphysical
models of the world (cognitive map) with the physical world, and communicate
them with other humans.

Humans are storytelling people. A popular historian, Yuval N. Harari, says that
our ability to tell stories, both fictional and factual, are one of the most
important attributes that set us apart from other animals during our cognitive
evolution, more than \todo{fact-check} 40.000 years ago. This ability to tell and
believe in stories made it possible for large groups of humans to cooperate, and
build the highly organized society we live in, in the modern day.

% NOTE:
% • The description of information science and computer technology. Exploration
% of using computers for mapping and spatial analysis.
% • Roger Tomlinson and the pioneering of GISc.
% • Advance of hardware and software for spatial analysis during the 70s and
% 80s.
% • Introduction of personal computers mad GIS tech more accessible.
% • 90s, integration with GPS and remote sensin. Accurate and real-time data
% collection. The advent of the internet played a crutial role in the
% dissemination of geographic information. Web-based mapping, democratization of
% data (OSM).
% • 2000s, proliferation of web-based mapping services. Epidemiology,
% transportation planning etc. Government, business and academia.
% • Artificial intelligence, big data analytics and augmented reality.
